\documentclass[preprint]{sigplanconf}
%\usepackage[square, comma, sort&compress]{natbib}

\usepackage{amsmath}

\begin{document}

%\conferenceinfo{PLDI �09}{ todo }

%\copyrightyear{2005}

%\copyrightdata{1-59593-056-6/05/0006}

\preprintfooter{DRAFT}

%\titlebanner{DRAFT}

\title{Title}


\authorinfo{Double blind}
{ }

\maketitle

\begin{abstract}
This is the text of the abstract.
\end{abstract}

\category{CR-number}{subcategory}{third-level}

\terms
term1, term2

\keywords
keyword1, keyword2

\section{Introduction}
  Ynot is the right design from a modularity, re-use perspective.  Much of the success hinges on the use of a higher-order dependently-typed language (Coq) and the ability to smoothly integrate modeling (inductive definitions), domain-specific abstractions (e.g., STsep) and uniform abstraction (Pi). If Ynot is to succeed, need drastic improvements in automation.
  \subsection{Summary of pain from ICFP paper}
  \subsection{Contributions}
    Dramatic reduction in cost of writing/maintaining Ynot code:
    \begin{itemize}
    \item new formulation of HTT in Coq (specifically STsep)
    \item general higher-order tactic for discharging STsep VCs (sep)
    \item how how domain-specific predicates (e.g., listseg, itersep, etc.) can be integrated to keep verification burden relatively low.
    \item careful proving ``style'' avoids sensitivities that make proof
       maintenance difficult.
    \item comparison against previous work and work of others.
 \end{itemize}

\section{Overview of HTT and previous version of Ynot}
   Condensed version of ICFP paper.

\section{Ynot in practice}
  Demonstrate problems with binary post-conditions (including need for precision), size of old proofs, difficulty in maintaining them.

\section{New version of Ynot}
 Perhaps use association list as running example?
  \subsection{Re-formulate STsep to use only unary heap predicates}
     key advantage: can use traditional separation-logic reasoning.  Things to include:
     \begin{enumerate}
     \item list some of the key theorems
     \item argue that these will be strung together below with Ltac support
     \item here's where we have to fudge the issue of ``ghost'' values and the need to add an injectivity axiom.
     \item another price is that code needs more annotations (but we attempt to minimize)
     \end{enumerate}
   \subsection{new tactic library - sep}
     Explain structure of sep, sketch how it's coded, and what it can do.

\section{Imperative ADTs in Ynot 2.0}
    Focus on extensibility -- e.g., adding itersep to support arrays, adding listseg to support linked lists, etc. and showing how these get used in ADTs like the queue, hash-table, etc.

\section{Comparison}
 \subsection{comparison to old Ynot ADT implementations}
  Some of this would happen above too.
 \subsection{comparison to Jahob}
 \subsection{comparison to Smallfoot}

\section{Other related work}
\section{Summary and conclusions}
Check that bibliography is working~\cite{1159812}.

%\bibliographystyle{plainnat}
\bibliographystyle{plain}

\bibliography{bib}

\end{document}