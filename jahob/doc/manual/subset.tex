\section{Java Subset}
\label{sec:javaSubset}

The goal of Jahob is to be a sound data structure consistency
verification tool for a subset of Java that is reasonably easy to
describe.\footnote{In principle, any unsound tool is sound on some
implicitly defined set of programs, defined as ``the programs for
which the tool is sound for'', but we would like to have an
independent and natural characterization of the class of Java
programs.}  The process of choosing features to analyze is similar to
language design---but stripped off the superficial issues such as
concrete syntax.

Jahob does not support the following features:
\begin{enumerate}
\item Anything beyond Java 1.4, specifically generics (to start with,
the Joust parser only supports ``Java Language Specification, 2nd Edition'').
\item Dynamic class loading.
\item Concurrency.
\item Exceptions.
\item Ad-hoc overloading (methods named the same in the same class).
\item Dynamic dispatch.
\end{enumerate}
Some of these features would be easier to handle, some would be more
difficult.  However, we were successful in writing programs that
manipulate highly non-trivial data structures in this subset, and we
have found that the use of Java subset makes the process easier than
the use of a specialized language.  The features that we choose not to
handle from Java seem largely orthogonal to the main goal of Jahob:
data structure consistency verification.

There are some additional restrictions on the use of
existing constructs, which are mostly either result of
keeping the specification methodology simple, or a result of
the current status of the implementation:
\begin{enumerate}
\item Complex field initializers: it is possible to initialize
a field to a simple constant value, but not to a value of some
compound expression.  In particular, this ensures that the initialization
does not require executing any code that may have side-effects.
\item The only currently supported loop construct is \q{while},
     the following control-flow constructs are \emph{not} supported: 
     \q{break}, \q{continue}, \q{do}, \q{goto}.
\end{enumerate}

On a positive side, Jahob supports:
\begin{enumerate}
\item Dynamically linked data structures.
\item Arrays.
\item Certain uses of interfaces.
\item Simple field initializers.
\item Constructors.
\end{enumerate}

Finally, Jahob introduces a range of new specification constructs; we
describe these constructs in the sequel.  These constructs are written
as special Java comments that do not interfere with normal compilation
using an ordinary Java compiler, but help the Jahob tool in the
verification task.
