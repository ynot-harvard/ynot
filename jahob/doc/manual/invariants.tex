\section{Class Invariants in Jahob}

Consider a class $C$ that implements some data structure.  A class
invariant of $C$ is a property of the data structure that holds ``most
of the time'' or ``when a class is in consistent state''.  Class
invariant approximates the set of states that are reachable after
executing any finite number of operations of that class.  To ensure
inductively that the invariants hold, we require that they hold when
the variables have their initial values, and require that they are
preserved by each public method of the class.  The public methods then
assume that the invariants hold at their entry.  Finally, because of
reentrant calls, the invariant is also assumed at all method calls to
methods outside the class and the public methods of the same class, so
that it can be assumed at the entry of those methods.  This simple
policy is very general, especially in the light of the ability to add
uninterpreted abstract ``guard'' variables that can make the truth
value of the invariant conditional.

Define an \emph{external call} as a call to a method outside the
current class or a call to a public method of the same class.  Also,
when we say \emph{method}, we include the constructors.  This means
that constructors are not special from the viewpoint of the validity
of invariants.

To summarize, when a class $C$ contains an invariant $I$, Jahob
takes the following actions:
\begin{itemize}
\item Jahob verifies that the invariant is true for the initial state
of the class.  To make sure that the invariant is true in the initial
state, the developer may wish to introduce auxiliary variables.
\item Jahob asserts the invariant (that is, requires the invariant 
to be true) at the following places:
\begin{itemize}
\item after the last statement in a public method of the class
\item at the return statement
\item before an external call
\end{itemize}
\item Jahob assumes the invariant at the following places:
\begin{itemize}
\item the entry of methods
\item the point after an external call
\end{itemize}
\end{itemize}
Note that, when we say ``the place before procedure call'' we are
referring to the place after evaluating the arguments and performing
any side-effects that this evaluation may require, and before the
actual procedure invocation.  This location is more clearly visible in
the Jahob Ast intermediate representation of the input program.

When an object is created, all fields are initialized either
to the specified simple value or to the default value
otherwise.  Default value for objects is null, for integers
0, for booleans false.

Hence, when an object of class T is created, this is a
sideffecting operation that modifies \q{T.alloc} and
initializes object.  Constructor call is a separate
statement, automatically generated in the intermediate
language from Java new statement.

Note that public constructors should thus also get to assume invariant
and preserve at the end.  They will typically differ in that they will
set the value of a boolean guard of an invariant, thus making the
truth value of the invariant be non-trivial.
