\section{Formulas}

\paragraph{Uses of formulas.}
Formulas are used in the following places:
\begin{itemize}
\item procedure contracts: preconditions, modifies clauses, postconditions
\item class invariants
\item definitions of specification variables
\item initial values of specification variables
\item loop invariants
\item \q{assume}, \q{assert}, and \q{split} statement
\end{itemize}

\paragraph{Free variables of formuals.}
The purpose of formulas is to express the relationships between values
in the program.  These values are denoted by free variables of the
formula.  In general, formulas contain two kinds of variables:
\begin{itemize}
\item bound variables (in quantifiers, lambda expressions, and set
comprehension)
\item free variables
\end{itemize}
As is standard in Hoare logic approaches, the free variables of the
formula denote the state variables of the program.  Therefore, a
formula may refer only to the free variables that are visible in the
scope where the formula is used.  For example, class invariants cannot
refer to procedure parameters, and procedure contracts cannot refer to
local variables of the procedure (but can refer to procedure
parameters).  In addition, procedure contracts of public methods can refer only to
\emph{class specification variables}, whereas procedure invariants and
the definitions of specification variables can refer only to
\emph{class imlementation variables}.  Procedure contracts and invariants in procedure bodies
can refer to to both specification and implementation variables; these
invariants apply to concrete state and specification variables can be
considered as a shorthand for their definition.  Class implementation
variables of the class $C$ consist of variables declared \q{private}
in $C$, variables declared $\q{claimedby}\, C$ in classes other than
$C$, and
\q{readonly} variables of $C$.  Class specification variables of the
class $C$ consist of variables not declared \q{private} in $C$, and
specification variables introduced in $C$ using \q{specfield} and
\q{specstatic} declarations, and \q{readonly} variables of $C$.
Note that \q{readonly} variables are both implementation and
specification variables.  Initial values of specification variables
can have no free variables, because they denote constants.

\paragraph{Syntax and typing of formulas.}
Formulas used in Jahob are a subset of Isabelle/HOL formulas
\cite{NipkowETAL02IsabelleHOL}.  Consequently, the formulas use
the syntax of lambda calculus, with the application of function $f$ to
arguments $x_1,\ldots,x_n$ denoted $f\, x_1\, \ldots x_n$.  We also
use some infix operators and binding operators \q{ALL} (for all),
\q{EX} (exists), $\q{\%}$ (lambda), and $\{ x. F \}$ (set
comprehension).  Formulas are typed with a simple type system that is
currently monomorphic, with some operators overloaded.  
Table~\ref{tab:formulaConstructs} gives an overview of the main constructs in formulas.

\begin{table*}
\begin{center}
\begin{tabular}{c|c|c|c|c}
construct & symbol & type & example & example meaning \\ \hline \hline
conjunction & \q{\&} & prop $\Rightarrow$ prop $\Rightarrow$ prop & P \q{\&} Q & both P and Q hold \\
disjunction & \q{|}  & prop $\Rightarrow$ prop $\Rightarrow$ prop & P \q{|} Q & P or Q hold (or both) \\
negation & ${\sim}$ & prop $\Rightarrow$ prop & ${\sim}$ P & P does not hold \\
implication & \q{-->}  & prop $\Rightarrow$ prop $\Rightarrow$ prop & P \q{-->} Q & if P holds, so does Q \\
equivalence & \q{=}  & prop $\Rightarrow$ prop $\Rightarrow$ prop & P \q{=} Q & P holds if and only if Q holds \\ \hline
universal quantifier & ALL & ('a $\Rightarrow$ prop) $\Rightarrow$ prop & ALL x::obj. F & 
        \parbox[c]{1.5in}{for all elements x of type obj, formula F holds} \mnl
existential quantifier & EX & ('a $\Rightarrow$ prop) $\Rightarrow$ prop & EX x::obj. F & 
        \parbox[c]{1.5in}{there exists an element x of type obj for which F holds} \\ \hline
equality & = & 'a $\Rightarrow$ 'a $\Rightarrow$ prop & x=y & 
        \parbox[c]{1.5in}{x and y are equal values} \\ 
disequality & {${\sim}$=} & 'a $\Rightarrow$ 'a $\Rightarrow$ prop & x{${\sim}$=}y & 
        \parbox[c]{1.5in}{x and y are distinct values} \\ 
function application & f x & ('a $\Rightarrow$ 'b) $\Rightarrow$ 'a $\Rightarrow$ 'b & \q{next}\, x &
        \parbox[c]{1.5in}{value of function \q{next} for argument x} \\
anonymous functions & \q{\%} & & \q{\%} x. x + 1 & function that, given x, returns x+1 \\ \hline
tuple & ( , \ldots , ) & \parbox[c]{1in}{'a$_1$ $\Rightarrow$ \ldots $\Rightarrow$ 'a$_n$ $\Rightarrow$ 'a$_1$ * \ldots * 'a$_n$} & 
        (x,y) & the pair of values x and y \mnl
list constant & [ ; \ldots ; ] & 'a $\Rightarrow$ \ldots 'a $\Rightarrow$ 'a list & 
        [a;b;c] & the list containing elements a, b, c \mnl
set constant & \{ , \ldots , \} & 'a $\Rightarrow$ \ldots 'a $\Rightarrow$ 'a set & 
        \{a, b, c\} & the set containing elements a, b, c \mnl
set comprehension & \{\ .\ \ \ \} & ('a $\Rightarrow$ prop) $\Rightarrow$ 'a set & \{p. EX x. p=(x, f x)\} &
        \parbox[c]{1.5in}{the set of (argument,value) pairs for function f} \\
member & : & 'a $\Rightarrow$ 'a set $\Rightarrow$ prop 
        & (x,y) : r & \parbox[c]{1.6in}{x is related to y in relation r} \\
not member & ${\sim}$: & 'a $\Rightarrow$ 'a set $\Rightarrow$ prop 
        & ALL y. (x,y) ${\sim}$: r & \parbox[c]{1.7in}{no element is related to x in r} \\
subset & \q{<=} & 'a set $\Rightarrow$ 'a set $\Rightarrow$ prop
        & \{x,y\} \q{<=} A & \parbox[c]{1.6in}{both x and y are in set A} \\
union & Un & 'a set $\Rightarrow$ 'a set $\Rightarrow$ 'a set
        & C \q{<=} A Un B & \parbox[c]{1.6in}{C is subset of the union of A and B} \\
intersection & Int & 'a set $\Rightarrow$ 'a set $\Rightarrow$ 'a set
        & A Int B = \{\} & \parbox[c]{1.6in}{A and B are disjoint} \\
set difference & $-$ & 'a set $\Rightarrow$ 'a set $\Rightarrow$ 'a set
        & A = B $-$ \{x\} & \parbox[c]{1.5in}{set A is the result of removing element x from set B} \\\hline
integer addition & $+$ & int $\Rightarrow$ int $\Rightarrow$ int 
        & x + y = y + x & addition is commutative \\
integer subtraction & $-$ & int $\Rightarrow$ int $\Rightarrow$ int 
        & (x + y) - x = y & a form of cancellation law \\
integer multiplication & * & int $\Rightarrow$ int $\Rightarrow$ int 
        & EX y. x = 7 * y & x is divisible by 7 \\
order on integers & $<$ & int $\Rightarrow$ int $\Rightarrow$ prop 
        & 0 $<$ x * x + 1 & square plus one is positive \\
\end{tabular}
\end{center}
\caption{Basic formula constructs\label{tab:formulaConstructs}}
\end{table*}

\begin{table*}
\begin{center}
\begin{tabular}{c|c|c|c|c}
construct & symbol & type & example & example meaning \\ \hline \hline
field dereference & .. & obj $\Rightarrow$ 'a field $\Rightarrow$ 'a 
        & x..next & the value of field next of object x \\
field update & ( := ) & 'a field $\Rightarrow$ obj $\Rightarrow$ 'a $\Rightarrow$ 'a field
        & z..(next(x:=y)) & \parbox[c]{2in}{the value of field next of object z after assignment x.next=y} \mnl
array lookup & .[\ ] & obj $\Rightarrow$ int $\Rightarrow$ value
        & a.[2] & the third element of array object a \\
variable location & @ & name $\Rightarrow$ loc
	& @count & location storing variable count \\
field location &  @.  & obj $\Rightarrow$ name $\Rightarrow$ loc 
	& x@.next & location storing 'next' field of x \\
array location & @[\ ]  & obj $\Rightarrow$ int $\Rightarrow$ loc 
	& a@[3] & the 4th cell of array a \\
\end{tabular}
\end{center}
\caption{Constructs for describing program state\label{tab:formulaProgram}}
\end{table*}

Table~\ref{tab:formulaProgram} gives an overview of constructs
specific to Jahob.  Formal definitions of these constructs are
provided in Jahob background theory file \q{lib/Jahob.thy}.  Fields
are treated as functions, so field dereference $x..f$ is a function
application $f x$.  Therefore, field update, which is useful for
expressing updates of fields, is represented by function update.
Similarly, array access is denoted by $.[]$.  In addition to denoting
\emph{values} of fields and array locations, it is useful to denote
their locations.  We use $@.$ to denote the location corresponding to
a field, and $@[]$ to denote location corresponding to an array.

\paragraph{The construct $\q{old}$.}
The expression $\q{old}\, x$ denotes the value of variable
$x$ at the beginning of the current procedure.  This
construct is meaningless in $\q{requires}$ and
$\q{modifies}$ clauses because thay are already interpreted
with respect to the initial state of the procedure.

Note that $\q{old}\, E$ where $E$ is an expression is merely
a syntactic shorthand for $\q{old}$ applied to each variable
occuring in $E$.  In particular, $\q{old}$ does \emph{not} satisfy
the fundamental property of equality $x = y\ \mathop{\q{-->}}\
\q{old}\, x = \q{old}\, y$, so it is not simply a built-in function
symbol.
