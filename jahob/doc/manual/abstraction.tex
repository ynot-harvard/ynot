\section{Data Abstraction in Jahob}

The main idea is to do refinement at the class level,
not at the object level.

For convenience, the representation specification can
refer to this, we take union of all representations over
all objects.  There is also a static representation specification,
which may not mention 'this'.


Locations are given by a set of
(object,field) pairs (and array regions?).
The cool thing is that the range of representation may change dynamically (!)

Specification variables.

Interfaces and classes.  Interface ownership condition for
interfaces with static fields: if an interface contains a
static variable, it is implemented by at most one class;
this class is called the owner of the interface.

Representation exposure and global specification of refinement
that allows object collaboration.

Variables without definition.  Public and abstract.  Abstract
statements in method bodies can modify them.

Constdefs: public and private.  Constdefs are pure and dependent, they
are logical shorthands.

Public invariants can be free and cross-module.  You can tell them
apart by checking whether they mention any variables outside the module: if they
mention only variables of the module, they are free, otherwise they are cross-module.
Free invariants are guaranteed by the module and need not be checked,
they can be used for free.  Cross-module invariants must be maintained
by all modules that see them; they are one module's contributions to
global set of invariants.  (But not entirely global, at some point everything
can be claimed by one module.)

\paragraph{Representation locations.}
Representation locations are locations that store the internal
representation of a module.  There are two uses of representation
locations:
\begin{itemize}
\item the invariants of a class $C$ may only depend on the representation
locations of the class $C$;
\item when a procedure in $C.p$ is called, the set of locations that are
in the representation of $C$ at the time of the call is implicitly allowed
to be modified by $C.p$, and need not be reported in the modifies clause.
\end{itemize}
In Jahob, the representation set contains both fixed and dynamically
changing locations.  At any given state a location belongs to at most
one module.  There is no way for a module $D$ to ``take away'' the
representation locations of another module $C$, without $C$ first
releasing these representation locations.

The representation locations of class $C$ consist of
\begin{itemize}
\item private variables of $C$
\item public variables of all classes claimed by $C$
\item fields claimed by $C$
\item fields of all objects in the set $C.rep$
\end{itemize}

Once the control is outside a module, the value of $C.rep$, and
therefore the set of representation locations does not change.  To
ensure that representation invariants are not violated outside the
module, each procedure therefore ensures that, at the end of procedure
execution, the truth value of each invariant remains the same when a
location outside the representation set at the end of the
procedure is changed.
